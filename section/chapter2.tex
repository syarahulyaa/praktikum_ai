\chapter{Related Works}

Your related works, and your purpose and contribution which must be different as below.

\section{Fadila/1164072}
\subsection{Teori}
Penyelesaian Tugas Harian 3 ( No. 1-7 )
\begin{enumerate}
\item Binary Classification Dan Ilustrasi Gambarnya
\begin{itemize}
\item Pengertian Binary Classification / Klasifikasi Biner:
\par Klasifikasi biner atau binomial merupakan tugas untuk mengklasifikasikan elemen-elemen dari himpunan tertentu ke dalam dua kelompok (memprediksi kelompok mana yang masing-masing dimiliki) ber-
\par dasarkan aturan klasifikasi.
\item Ilustrasi Gambar Binary Classification :
\par

\begin{figure}[ht]
\centering
\includegraphics[scale=0.3]{figures/binary.jpg}
\caption{binary classification}
\label{contoh}
\end{figure}

\par
\end{itemize}
\item Supervised Learning, Unsupervised Learning, Clustering Dan Ilustrasi Gambar
\begin{itemize}
\item Pengertian Supervised Learning :
\par Sebuah pendekatan dimana terdapat data yang dilatih dan ditargetkan. Leih singkatnya supervised learning memiliki kategori sehingga tujuan dan outputnya jelas.
\begin{itemize}
\par
\item Ilustrasi Gambar Supervised Learning :

\begin{figure}[ht]
\centering
\includegraphics[scale=0.4]{figures/supervised1.jpg}
\caption{supervised}
\label{contoh}
\end{figure}


\par Pada contoh gambar diatas dikelompokkan bahwa apabila bentuk dari objek di gambar berbentuk bundar dan berwarna merah maka akan dinamakan atau di sebut sebagai Apel. Dan apabila pada gambar terdapat objek berbentuk panjang dan berwarna kuning maka akan dinamakan atau di sebut sebagai pisang.
\par
\end{itemize}

\par
\item Pengertian Unsupervised Learning :
\par Tidak memiliki data latih, sehingga dari data yang tersebut kita bisa mengelompokkannya ke berbagai kelompok 2 seterusnya. Dengan lebih singkatnya ialah unsupervised learning tidak memiliki kategori.
\par
\par
\begin{itemize}
\item Ilustrasi Gambar Unsupervised Learning :

\begin{figure}[ht]
\centering
\includegraphics[scale=0.4]{figures/unsupervised2.jpg}
\caption{unsupervised}
\label{contoh}
\end{figure}

\par Pada gambar dapat dilihat bahwa ada banyak data namun tidak pada pengelompokkan yang tepat. Cuman dapat dikelompokkan ke dalam berbagai macam bentuk dan jumlah namun tidak memberikan output yang jelas.
\par
\par
\end{itemize}
\item Pengertian Clustering :
\par Metode pengelompokan data. Clustering juga merupakan proses partisi satu set objek data ke dalam himpunan bagian yang disebut dengan cluster. Objek dalam cluster tersebut memiliki kemiripan karakteristik antar satu sama lain.
\par

\begin{figure}[ht]
\centering
\includegraphics[scale=0.5]{figures/clustering.jpg}
\caption{clustering}
\label{contoh}
\end{figure}

\par
\end{itemize}
\item Evaluasi, Akurasi Dan Ilustrasi Gambar
\begin{itemize}
\item Pengertian Evaluasi
\par Evaluasi digunakan untuk memeriksa/memastikan dan mengevaluasi model dalam bekerja ( seberapa baik ) dengan mengukur keakuratannya. Kita juga dapat menanalisis kesalahan yang dibuat pada model yang dijalankan, tingkat kebingungan dan menggunakan matriks kebingunan.
\begin{figure}[ht]
\centering
\includegraphics[scale=0.6]{figures/eva.jpg}
\caption{Evaluasi}
\label{contoh}
\end{figure}

\par Pada contoh gambar dapat dilihat bahwa dilakukan evaluasi terhadap kerja dalam penentuan jenis dari objek. Dievaluasi berapa banyak sebuah objek ketika dikelompokkan dan diklasifikasikan kemudian dapat dilihat apakah kerjanya sesuai atau tidak.
\par

\par
\item Pengertian Akurasi
\par Accuracy akan didefinisikan sebagai presentasi kasus yang diklasifikasikan dengan benar. Accuracy lebih jelasnya adalah perbandingan kasus yang diidentifikasi benar dengan jumlah semua kasus
\par Rumus dari accuracy= (a+c)/(a+b+c+d)
\par

\begin{figure}[ht]
\centering
\includegraphics[scale=0.8]{figures/acuracy.jpg}
\caption{Akurasi}
\label{contoh}
\end{figure}

\par Dilakukan perhitungan dengan rumus akurasi terhadap data yang telah diolah pada " Evaluasi ". Kemudian di dapatkan hasil dari pengolahan data tersebut.
\par Contoh penggabungan Akurasi Dan Evaluasi
\par

\begin{figure}[ht]
\centering
\includegraphics[scale=0.5]{figures/evacuray.jpg}
\caption{Contoh Evaluasi Dan Akurasi Secara Bersamaan }
\label{contoh}
\end{figure}

\end{itemize}

\par
\item Membuat Dan Membaca Confusion Matrix Beserta Contoh
\begin{itemize}
\item Pengertian Confusion Matrix
\par Confusion matrix merupakan suatu metode yang digunakan untuk melakukan perhitungan akurasi pada konsep data mining.
\item Pembacaan Confusion Matrix
\begin{enumerate}
\item Apabila hasil prediksi negatif dan data sebenarnya merupakan negatif.
\item Apabila hasil prediksi positif sedangkan nilai sebenarnya merupakan negatif.
\item Apabila hasil prediksi negatif sedangkan nilai sebenarnya merupakan positif.
\item Apabila hasil prediksi positif dan nilai sebenarnya merupakan positif.
\end{enumerate}
\par
\par
\item Pembuatan Confusion Matrix
\begin{enumerate}
\item Menentukan 4 proses klasifikasi yang akan digunakan dalam confusion matrix.
\item 4 Istilah tersebut ada True Positive ( TP ), True Negative ( TN ), False Positive ( FP ) dan False Negative ( FN ).
\item Kelompokkan klasifikasi tersebut bisa menggunakan klasifikasi biner
\item Akan menghasilkan keluaran berupa 2 Kelas ( Positif dan Negatif ) dan penentuan TP, FP ( 1 klasifikasi positif ) , FN dan TN ( 1 klasifikasi negatif ).
\item Contoh dasarnya nampak seperti langkah diatas
\item Istilahnya daat didefinisikan dengan objek lain namu dengan alur yang sama ( sesuai rumus baik klasifikasi dll ).
\end{enumerate}
\par

\item Ilustrasi Gambar
\par

\begin{figure}[ht]
\centering
\includegraphics[scale=0.6]{figures/confusion.jpg}
\caption{confusion matrix}
\label{contoh}
\end{figure}

\par
\begin{itemize}
\item Penjelasan
\begin{enumerate}
\item Recall
\par Dari semua kelas positif, seberapa banyak yang kami prediksi dengan benar. Itu harus setinggi mungkin.
\par

\begin{figure}[ht]
\centering
\includegraphics[scale=0.6]{figures/recall.jpg}
\caption{recall}
\label{contoh}
\end{figure}

\par
\item Presisi / Precision
\par Dari semua kelas, seberapa banyak yang kami prediksi dengan benar. Itu harus setinggi mungkin.
\par

\begin{figure}[ht]
\centering
\includegraphics[scale=0.6]{figures/precision.jpg}
\caption{precision}
\label{contoh}
\end{figure}

\par
\item F-Ukur ( measure )
\par Sulit untuk membandingkan dua model dengan presisi rendah dan daya ingat tinggi atau sebaliknya. Jadi untuk membuatnya sebanding, kami menggunakan F-Score. F-score membantu mengukur Recall dan Precision pada saat yang bersamaan
\par

\begin{figure}[ht]
\centering
\includegraphics[scale=0.6]{figures/f.jpg}
\caption{f-measure}
\label{contoh}
\end{figure}

\par
\end{enumerate}
\end{itemize}
\end{itemize}


\par
\item Cara Kerja K-Fold Classification Dan Ilustrasi Gambar
\begin{enumerate}
\item Pertama-tama untuk total instance dibagi menjadi N bagian.
\item Fold ke-1 ( atau pertama ) adalah ketika bagian ke-1 menjadi data uji (testing data) dan sisanya menjadi data latih (training data).
\item Hitung akurasi ( berdasarkan porsi data tersebut. Persamaanya sebagai berikut :
\par (sigma) data klasifikasi
\par (sigma) total data uji
\par x 100 persen 
\item Fold ke-2 ( kedua ) adalah ketika bagian ke-2 menjadi data uji (testing data) dan sisanya menjadi data latih (training data). 
\item Kemudian dihitunglah akurasi berdasarkan porsi data yang telah ditentukan
\item Demikian seterusnya hingga mencapai fold ke-K. Hitung rata-rata akurasi dari K buah akurasi di atas. Rata-rata akurasi ini menjadi akurasi final atau akhir.
\end{enumerate}
\par
\begin{itemize}
\item Ilustrasi Gambar
\par

\begin{figure}[ht]
\centering
\includegraphics[scale=0.4]{figures/hasilk1.jpg}
\caption{k-fold classification 1}
\label{contoh}
\end{figure}

\begin{figure}[ht]
\centering
\includegraphics[scale=0.4]{figures/hasilk2.jpg}
\caption{k-fold classification 2}
\label{contoh}
\end{figure}

\par
\par
\end{itemize}

\par
\item Decision Tree Dan Ilustrasi Gambar
\begin{itemize}
\item Pengertian Decision Tree
\par Decision tree adalah salah satu metode klasifikasi yang paling populer karena mudah diinterpretasikan oleh manusia. Decision tree merupakan metode klasifikasi yang digunakan untuk pengenalan pola dan termasuk dalam pengenalan pola secara statistik. 3 tipe dari decision tree ialah: simpul: simpul root, simpul perantara, dan simpul leaf.
\par

\end{itemize}
\par

\par
\begin{itemize}
\item Ilustrasi Gambar
\par


\begin{figure}[ht]
\centering
\includegraphics[scale=0.5]{figures/decisiontree.jpg}
\caption{decision tree}
\label{contoh}
\end{figure}

\par
\end{itemize}
\item Information Gain Dan Entropi
\begin{itemize}
\item Pengertian Information Gain
\par Information Gain adalah salah satu atribute selection measure yang digunakan untuk memilih test atribute tiap node pada tree.
\par Algoritme Information Gain digunakan untuk mengurangi dimensi atribut untuk mendapatkan atribut-atribut yang relevan. 
\par

\begin{itemize}
\item Ilustrasi Gambar
\par


\begin{figure}[ht]
\centering
\includegraphics[scale=0.6]{figures/information1.jpg}
\caption{informaion gain 1}
\label{contoh}
\end{figure}


\begin{figure}[ht]
\centering
\includegraphics[scale=0.6]{figures/information2.jpg}
\caption{information gain 2}
\label{contoh}
\end{figure}

\item Penjelasan :
\par Tabel 1 sampai dengan Tabel 2 menunjukkan bahwa penggunaan seleksi fitur Information Gain menghasilkan nilai akurasi yang lebih baik dibandingkan tanpa menggunakan Information Gain. 
\par Pada saat nilai K sama dengan 5 ( K=5) akurasi yang dihasilkan sistem tanpa menggunakan Information Gain menunjukkan hasil yang kurang baik pada sebaran kelas seimbang maupun tak seimbang yaitu 61,54 persen pada sebaran kelas seimbang dan 73,08 persen pada sebaran kelas tidak seimbang. 
\end{itemize}


\par
\item Pengertian Entropi
\par Entropi pada umumnya merupakan salah satu besaran yang mengukur energi dalam sistem per satuan temperatur yang tak dapat digunakan untuk melakukan usaha.
\par Namun, secara spesifik untuk " Entropi " sendiri merupakan parameter untuk mengukur tingkat keberagaman (heterogenitas) dari kumpulan data. 

\par

\begin{figure}[ht]
\centering
\includegraphics[scale=0.5]{figures/entropi.jpg}
\caption{entropi}
\label{contoh}
\end{figure}

\par
\end{itemize}

\end{enumerate}

\section{Lusia Violita Aprilian}

\subsection{binary classification dilengkapi ilustrasi gambar}

\begin{enumerate}

\item Binary classification yaitu berupa kelas positif dan kelas negatif. Klasifikasi biner adalah dikotomisasi yang diterapkan untuk tujuan praktis, dan dalam banyak masalah klasifikasi biner praktis, kedua kelompok tidak simetris - daripada akurasi keseluruhan, proporsi relatif dari berbagai jenis kesalahan yang menarik. Misalnya, dalam pengujian medis, false positive (mendeteksi penyakit ketika tidak ada) dianggap berbeda dari false negative (tidak mendeteksi penyakit ketika hadir).

\begin{figure}[ht]
\centering
\includegraphics[scale=0.5]{figures/f1.jpg}
\caption{Binary Classification}
\label{contoh}
\end{figure}

\subsection{supervised learning dan unsupervised learning dan clustering\\ dengan ilustrasi gambar}

\item Supervised learning adalah tugas pembelajaran mesin untuk mempelajari suatu fungsi yang memetakan input ke output berdasarkan contoh pasangan input-output. Ini menyimpulkan fungsi dari data pelatihan berlabel yang terdiri dari serangkaian contoh pelatihan. Dalam pembelajaran yang diawasi, setiap contoh adalah pasangan yang terdiri dari objek input (biasanya vektor) dan nilai output yang diinginkan (juga disebut sinyal pengawas). Algoritma pembelajaran yang diawasi menganalisis data pelatihan dan menghasilkan fungsi yang disimpulkan, yang dapat digunakan untuk memetakan contoh-contoh baru. Skenario optimal akan memungkinkan algoritma menentukan label kelas dengan benar untuk instance yang tidak terlihat. Ini membutuhkan algoritma pembelajaran untuk menggeneralisasi dari data pelatihan untuk situasi yang tidak terlihat dengan cara yang "masuk akal" (lihat bias induktif). Tugas paralel dalam psikologi manusia dan hewan sering disebut sebagai pembelajaran konsep.

\begin{figure}[ht]
\centering
\includegraphics[scale=0.5]{figures/f2.jpg}
\caption{Supervised Learning}
\label{contoh}
\end{figure}

\item Unsupervised learning adalah istilah yang digunakan untuk pembelajaran bahasa Ibrani, yang terkait dengan pembelajaran tanpa guru, juga dikenal sebagai organisasi mandiri dan metode pemodelan kepadatan probabilitas input. Analisis cluster sebagai cabang pembelajaran mesin yang mengelompokkan data yang belum diberi label, diklasifikasikan atau dikategorikan. Alih-alih menanggapi umpan balik, analisis klaster mengidentifikasi kesamaan dalam data dan bereaksi berdasarkan ada tidaknya kesamaan di setiap potongan data baru.

\begin{figure}[ht]
\centering
\includegraphics[scale=0.5]{figures/f3.jpg}
\caption{Unsupervised Learning}
\label{contoh}
\end{figure}

\item Cluster analysis or clustering adalah tugas pengelompokan sekumpulan objek sedemikian rupa sehingga objek dalam kelompok yang sama (disebut klaster) lebih mirip (dalam beberapa hal) satu sama lain daripada pada kelompok lain (kluster). Ini adalah tugas utama penambangan data eksplorasi, dan teknik umum untuk analisis data statistik, yang digunakan di banyak bidang, termasuk pembelajaran mesin, pengenalan pola, analisis gambar, pengambilan informasi, bioinformatika, kompresi data, dan grafik komputer. Analisis Cluster sendiri bukan merupakan salah satu algoritma spesifik, tetapi tugas umum yang harus dipecahkan. Ini dapat dicapai dengan berbagai algoritma yang berbeda secara signifikan dalam pemahaman mereka tentang apa yang merupakan sebuah cluster dan bagaimana cara menemukannya secara efisien. Gagasan populer mengenai cluster termasuk kelompok dengan jarak kecil antara anggota cluster, area padat ruang data, interval atau distribusi statistik tertentu. Clustering karena itu dapat dirumuskan sebagai masalah optimasi multi-objektif. Algoritma pengelompokan dan pengaturan parameter yang sesuai (termasuk parameter seperti fungsi jarak yang akan digunakan, ambang kepadatan atau jumlah cluster yang diharapkan) tergantung pada set data individual dan penggunaan hasil yang dimaksudkan. Analisis kluster bukan merupakan tugas otomatis, tetapi proses berulang penemuan pengetahuan atau optimasi multi-objektif interaktif yang melibatkan percobaan dan kegagalan. Seringkali diperlukan untuk memodifikasi praproses data dan parameter model hingga hasilnya mencapai properti yang diinginkan.

\begin{figure}[ht]
\centering
\includegraphics[scale=0.5]{figures/f4.jpg}
\caption{Cluster}
\label{contoh}
\end{figure}

\subsection{evaluasi dan akurasi dari buku dan disertai ilustrasi contoh
dengan gambar}

\item Evaluasi adalah tentang bagaimana kita dapat mengevaluasi seberapa baik model bekerja dengan mengukur akurasinya. Dan akurasi akan didefinisikan sebagai persentase kasus yang diklasifikasikan dengan benar. Kita dapat menganalisis kesalahan yang dibuat oleh model, atau tingkat kebingungannya, menggunakan matriks kebingungan. Matriks kebingungan mengacu pada kebingungan dalam model, tetapi matriks kebingungan ini bisa menjadi sedikit sulit untuk dipahami ketika mereka menjadi sangat besar.

\begin{figure}[ht]
\centering
\includegraphics[scale=0.5]{figures/f9.jpg}
\caption{ Evaluasi dan Akurasi}
\label{contoh}
\end{figure}

\subsection{ bagaimana cara membuat dan membaca confusion matrix, buat confusion matrix }

\item Cara membuat dan membaca confusion matrix :
\begin{itemize}
\item 1) Tentukan pokok permasalahan dan atributanya, misal gaji dan listik.
\item 2) Buat pohon keputusan
\item 3) Lalu data testingnya
\item 4) Lalu mencari nilai a, b, c, dan d. Semisal a = 5, b = 1, c = 1, dan d = 3.
\item 5) Selanjutnya mencari nilai recall, precision, accuracy, serta dan error rate.
\end{itemize}
\item Berikut adalah contoh dari confusion matrix :
\begin{itemize}
\item Recall =3/(1+3) = 0,75
\item Precision = 3/(1+3) = 0,75
\item Accuracy =(5+3)/(5+1+1+3) = 0,8
\item Error Rate =(1+1)/(5+1+1+3) = 0,2
\end{itemize}

\subsection{bagaimana K-fold cross validation bekerja dengan gambar ilustrasi}

\item Cara kerja K-fold cross validation :
\begin{itemize}
\item 1) Total instance dibagi menjadi N bagian.
\item 2) Fold yang pertama adalah bagian pertama menjadi data uji (testing data) dan sisanya menjadi training data.
\item 3) Lalu hitung akurasi berdasarkan porsi data tersebut dengan menggunakan persamaan.
\item 4) Fold yang ke dua adalah bagian ke dua menjadi data uji (testing data) dan sisanya training data. 
\item 5) Kemudian hitung akurasi berdasarkan porsi data tersebut.
\item 6) Dan seterusnya hingga habis mencapai fold ke-K.
\item 7) Terakhir hitung rata-rata akurasi K buah.
\end{itemize}

\begin{figure}[ht]
\centering
\includegraphics[scale=0.5]{figures/f5.jpg}
\caption{K-fold cross validation }
\label{contoh}
\end{figure}

\subsection{decision tree dengan gambar ilustrasi}

Decision tree adalah model visual yang terdiri dari node dan cabang, seperti Gambar dijelaskan secara rinci nanti dalam artikel ini. Untuk saat ini, amati bahwa ia tumbuh dari kiri ke kanan, dimulai dengan simpul keputusan root (kuadrat, juga disebut simpul pilihan) yang cabang-cabangnya mewakili dua atau lebih opsi bersaing yang tersedia bagi para pembuat keputusan. Pada akhir cabang awal ini, ada simpul akhir (segitiga, juga disebut simpul nilai) atau simpul ketidakpastian (lingkaran, juga disebut simpul peluang). Node akhir mewakili nilai tetap. Cabang lingkaran mewakili hasil yang mungkin bersama dengan probabilitasnya masing-masing (yang berjumlah 1,0). Di luar cabang-cabang node ketidakpastian awal ini, mungkin ada lebih banyak bujur sangkar dan lebih banyak lingkaran, yang umumnya bergantian sampai setiap jalur berakhir di simpul akhir.

\begin{figure}[ht]
\centering
\includegraphics[scale=0.5]{figures/f6.jpg}
\caption{Decision Tree}
\label{contoh}
\end{figure}

\subsection{Information gain dan entropi dengan gambar ilustrasi}

\item Information gain (IG) mengukur seberapa banyak informasi fitur memberi kita tentang kelas. - Fitur yang sempurna mempartisi harus memberikan informasi maksimal. - Fitur yang tidak terkait seharusnya tidak memberikan informasi.

\begin{figure}[ht]
\centering
\includegraphics[scale=0.5]{figures/f7.jpg}
\caption{Information gain}
\label{contoh}
\end{figure}

\item Entropi merupakan kemurnian dalam koleksi contoh yang sewenang-wenang.

\begin{figure}[ht]
\centering
\includegraphics[scale=0.5]{figures/f8.jpg}
\caption{Entropi}
\label{contoh}
\end{figure}

\subsection{binary classification dilengkapi ilustrasi gambar}

\item Binary classification yaitu berupa kelas positif dan kelas negatif. Klasifikasi biner adalah dikotomisasi yang diterapkan untuk tujuan praktis, dan dalam banyak masalah klasifikasi biner praktis, kedua kelompok tidak simetris - daripada akurasi keseluruhan, proporsi relatif dari berbagai jenis kesalahan yang menarik. Misalnya, dalam pengujian medis, false positive (mendeteksi penyakit ketika tidak ada) dianggap berbeda dari false negative (tidak mendeteksi penyakit ketika hadir).

\begin{figure}[ht]
\centering
\includegraphics[scale=0.5]{figures/f1.jpg}
\caption{Binary Classification}
\label{contoh}
\end{figure}

\subsection{supervised learning dan unsupervised learning dan clustering\\ dengan ilustrasi gambar}

\item Supervised learning adalah tugas pembelajaran mesin untuk mempelajari suatu fungsi yang memetakan input ke output berdasarkan contoh pasangan input-output. Ini menyimpulkan fungsi dari data pelatihan berlabel yang terdiri dari serangkaian contoh pelatihan. Dalam pembelajaran yang diawasi, setiap contoh adalah pasangan yang terdiri dari objek input (biasanya vektor) dan nilai output yang diinginkan (juga disebut sinyal pengawas). Algoritma pembelajaran yang diawasi menganalisis data pelatihan dan menghasilkan fungsi yang disimpulkan, yang dapat digunakan untuk memetakan contoh-contoh baru. Skenario optimal akan memungkinkan algoritma menentukan label kelas dengan benar untuk instance yang tidak terlihat. Ini membutuhkan algoritma pembelajaran untuk menggeneralisasi dari data pelatihan untuk situasi yang tidak terlihat dengan cara yang "masuk akal" (lihat bias induktif). Tugas paralel dalam psikologi manusia dan hewan sering disebut sebagai pembelajaran konsep.

\begin{figure}[ht]
\centering
\includegraphics[scale=0.5]{figures/f2.jpg}
\caption{Supervised Learning}
\label{contoh}
\end{figure}

\item Unsupervised learning adalah istilah yang digunakan untuk pembelajaran bahasa Ibrani, yang terkait dengan pembelajaran tanpa guru, juga dikenal sebagai organisasi mandiri dan metode pemodelan kepadatan probabilitas input. Analisis cluster sebagai cabang pembelajaran mesin yang mengelompokkan data yang belum diberi label, diklasifikasikan atau dikategorikan. Alih-alih menanggapi umpan balik, analisis klaster mengidentifikasi kesamaan dalam data dan bereaksi berdasarkan ada tidaknya kesamaan di setiap potongan data baru.

\begin{figure}[ht]
\centering
\includegraphics[scale=0.5]{figures/f3.jpg}
\caption{Unsupervised Learning}
\label{contoh}
\end{figure}

\item Cluster analysis or clustering adalah tugas pengelompokan sekumpulan objek sedemikian rupa sehingga objek dalam kelompok yang sama (disebut klaster) lebih mirip (dalam beberapa hal) satu sama lain daripada pada kelompok lain (kluster). Ini adalah tugas utama penambangan data eksplorasi, dan teknik umum untuk analisis data statistik, yang digunakan di banyak bidang, termasuk pembelajaran mesin, pengenalan pola, analisis gambar, pengambilan informasi, bioinformatika, kompresi data, dan grafik komputer. Analisis Cluster sendiri bukan merupakan salah satu algoritma spesifik, tetapi tugas umum yang harus dipecahkan. Ini dapat dicapai dengan berbagai algoritma yang berbeda secara signifikan dalam pemahaman mereka tentang apa yang merupakan sebuah cluster dan bagaimana cara menemukannya secara efisien. Gagasan populer mengenai cluster termasuk kelompok dengan jarak kecil antara anggota cluster, area padat ruang data, interval atau distribusi statistik tertentu. Clustering karena itu dapat dirumuskan sebagai masalah optimasi multi-objektif. Algoritma pengelompokan dan pengaturan parameter yang sesuai (termasuk parameter seperti fungsi jarak yang akan digunakan, ambang kepadatan atau jumlah cluster yang diharapkan) tergantung pada set data individual dan penggunaan hasil yang dimaksudkan. Analisis kluster bukan merupakan tugas otomatis, tetapi proses berulang penemuan pengetahuan atau optimasi multi-objektif interaktif yang melibatkan percobaan dan kegagalan. Seringkali diperlukan untuk memodifikasi praproses data dan parameter model hingga hasilnya mencapai properti yang diinginkan.

\begin{figure}[ht]
\centering
\includegraphics[scale=0.5]{figures/f4.jpg}
\caption{Cluster}
\label{contoh}
\end{figure}

\subsection{evaluasi dan akurasi dari buku dan disertai ilustrasi contoh\\ dengan gambar}

\item Evaluasi adalah tentang bagaimana kita dapat mengevaluasi seberapa baik model bekerja dengan mengukur akurasinya. Dan akurasi akan didefinisikan sebagai persentase kasus yang diklasifikasikan dengan benar. Kita dapat menganalisis kesalahan yang dibuat oleh model, atau tingkat kebingungannya, menggunakan matriks kebingungan. Matriks kebingungan mengacu pada kebingungan dalam model, tetapi matriks kebingungan ini bisa menjadi sedikit sulit untuk dipahami ketika mereka menjadi sangat besar.

\begin{figure}[ht]
\centering
\includegraphics[scale=0.5]{figures/f9.jpg}
\caption{ Evaluasi dan Akurasi}
\label{contoh}
\end{figure}

\subsection{ bagaimana cara membuat dan membaca confusion matrix, buat confusion matrix }

\item Cara membuat dan membaca confusion matrix :
\begin{itemize}
\item 1) Tentukan pokok permasalahan dan atributanya, misal gaji dan listik.
\item 2) Buat pohon keputusan
\item 3) Lalu data testingnya
\item 4) Lalu mencari nilai a, b, c, dan d. Semisal a = 5, b = 1, c = 1, dan d = 3.
\item 5) Selanjutnya mencari nilai recall, precision, accuracy, serta dan error rate.
\end{itemize}
\item Berikut adalah contoh dari confusion matrix :
\begin{itemize}
\item Recall =3/(1+3) = 0,75
\item Precision = 3/(1+3) = 0,75
\item Accuracy =(5+3)/(5+1+1+3) = 0,8
\item Error Rate =(1+1)/(5+1+1+3) = 0,2
\end{itemize}

\subsection{bagaimana K-fold cross validation bekerja dengan gambar ilustrasi}

\item Cara kerja K-fold cross validation :
\begin{itemize}
\item 1) Total instance dibagi menjadi N bagian.
\item 2) Fold yang pertama adalah bagian pertama menjadi data uji (testing data) dan sisanya menjadi training data.
\item 3) Lalu hitung akurasi berdasarkan porsi data tersebut dengan menggunakan persamaan.
\item 4) Fold yang ke dua adalah bagian ke dua menjadi data uji (testing data) dan sisanya training data. 
\item 5) Kemudian hitung akurasi berdasarkan porsi data tersebut.
\item 6) Dan seterusnya hingga habis mencapai fold ke-K.
\item 7) Terakhir hitung rata-rata akurasi K buah.
\end{itemize}
\begin{figure}[ht]
\centering
\includegraphics[scale=0.5]{figures/f5.jpg}
\caption{K-fold cross validation }
\label{contoh}
\end{figure}

\subsection{decision tree dengan gambar ilustrasi}

\item Decision tree adalah model visual yang terdiri dari node dan cabang, seperti Gambar dijelaskan secara rinci nanti dalam artikel ini. Untuk saat ini, amati bahwa ia tumbuh dari kiri ke kanan, dimulai dengan simpul keputusan root (kuadrat, juga disebut simpul pilihan) yang cabang-cabangnya mewakili dua atau lebih opsi bersaing yang tersedia bagi para pembuat keputusan. Pada akhir cabang awal ini, ada simpul akhir (segitiga, juga disebut simpul nilai) atau simpul ketidakpastian (lingkaran, juga disebut simpul peluang). Node akhir mewakili nilai tetap. Cabang lingkaran mewakili hasil yang mungkin bersama dengan probabilitasnya masing-masing (yang berjumlah 1,0). Di luar cabang-cabang node ketidakpastian awal ini, mungkin ada lebih banyak bujur sangkar dan lebih banyak lingkaran, yang umumnya bergantian sampai setiap jalur berakhir di simpul akhir.
\begin{figure}[ht]
\centering
\includegraphics[scale=0.5]{figures/f6.jpg}
\caption{Decision Tree}
\label{contoh}
\end{figure}

\subsection{Information gain dan entropi dengan gambar ilustrasi}

\item Information gain (IG) mengukur seberapa banyak "informasi" fitur memberi kita tentang kelas. - Fitur yang sempurna mempartisi harus memberikan informasi maksimal. - Fitur yang tidak terkait seharusnya tidak memberikan informasi.
\begin{figure}[ht]
\centering
\includegraphics[scale=0.5]{figures/f7.jpg}
\caption{Information gain}
\label{contoh}
\end{figure}

\item Entropi merupakan kemurnian dalam koleksi contoh yang sewenang-wenang.
\begin{figure}[ht]
\centering
\includegraphics[scale=0.5]{figures/f8.jpg}
\caption{Entropi}
\label{contoh}
\end{figure}
\end{enumerate}


\section{Rahmi Roza/1164085}
\subsection{Teori}
Penyelesaian Tugas Harian 3 ( No. 1-7 )
\begin{enumerate}
\item Binary Classification Dan Ilustrasi Gambarnya
\begin{itemize}
\item Pengertian Binary Classification / Klasifikasi Biner:
\par Klasifikasi biner atau binomial adalah tugas mengklasifikasikan elemen-elemen dari himpunan yang diberikan ke dalam dua kelompok (memprediksi kelompok mana yang masing-masing dimiliki) berdasarkan aturan klasifikasi.
\item Ilustrasi Gambar Binary Classification :
\par

\begin{figure}[ht]
\centering
\includegraphics[scale=0.3]{figures/binary.jpg}
\caption{binary classification}
\label{gambar1}
\end{figure}

\par
\item Supervised Learning, Unsupervised Learning, Clustering Dan Ilustrasi Gambar
\begin{itemize}
\item Pengertian Supervised Learning :
\par Supervised learning adalah sebuah pendekatan dimana sudah terdapat data yang dilatih, dan terdapat variable yang ditargetkan sehingga tujuan dari pendekatan ini adalah mengkelompokan suatu data ke data yang sudah ada.
\end{itemize}
\par
\item Ilustrasi Gambar Supervised Learning :

\begin{figure}[ht]
\centering
\includegraphics[scale=0.4]{figures/supervised1.jpg}
\caption{supervised}
\label{gambar2}
\end{figure}

\item Pengertian Unsupervised Learning :
\par unsupervised learning tidak memiliki data latih, sehingga dari data yang ada, kita mengelompokan data tersebut menjadi 2 bagian atau 3 bagian dan seterusnya.
\par
\par
\begin{itemize}
\item Ilustrasi Gambar Unsupervised Learning :

\begin{figure}[ht]
\centering
\includegraphics[scale=0.4]{figures/unsupervised2.jpg}
\caption{unsupervised}
\label{gambar3}
\end{figure}
\end{itemize}

\item Pengertian Clustering :
\par Metode pengelompokan data. Clustering juga merupakan proses partisi satu set objek data ke dalam himpunan bagian yang disebut dengan cluster. Objek dalam cluster tersebut memiliki kemiripan karakteristik antar satu sama lain.
\par

\begin{figure}[ht]
\centering
\includegraphics[scale=0.5]{figures/clustering.jpg}
\caption{clustering}
\label{gambar4}
\end{figure}

\par
\end{itemize}
\item Evaluasi, Akurasi Dan Ilustrasi Gambar
\begin{itemize}
\item Pengertian Evaluasi
\par Evaluasi digunakan untuk memeriksa/memastikan dan mengevaluasi model dalam bekerja ( seberapa baik ) dengan mengukur keakuratannya. Kita juga dapat menanalisis kesalahan yang dibuat pada model yang dijalankan, tingkat kebingungan dan menggunakan matriks kebingunan.
\par

\begin{figure}[ht]
\centering
\includegraphics[scale=0.6]{figures/eva.jpg}
\caption{Evaluasi}
\label{gambar5}
\end{figure}


\par
\item Pengertian Akurasi
\par Accuracy akan didefinisikan sebagai presentasi kasus yang diklasifikasikan dengan benar. Accuracy lebih jelasnya adalah perbandingan kasus yang diidentifikasi benar dengan jumlah semua kasus
\par Rumus dari accuracy= (a+c)/(a+b+c+d)
\par

\begin{figure}[ht]
\centering
\includegraphics[scale=0.8]{figures/acuracy.jpg}
\caption{Akurasi}
\label{contoh}
\end{figure}

\par Dilakukan perhitungan dengan rumus akurasi terhadap data yang telah diolah pada " Evaluasi ". Kemudian di dapatkan hasil dari pengolahan data tersebut.
\par Contoh penggabungan Akurasi Dan Evaluasi
\par

\begin{figure}[ht]
\centering
\includegraphics[scale=0.5]{figures/evacuray.jpg}
\caption{Contoh Evaluasi Dan Akurasi Secara Bersamaan }
\label{contoh}
\end{figure}

\end{itemize}

\par
\item Membuat Dan Membaca Confusion Matrix Beserta Contoh
\begin{itemize}
\item Pengertian Confusion Matrix
\par Confusion matrix adalah suatu metode yang biasanya digunakan untuk melakukan perhitungan akurasi pada konsep data mining. Rumus ini melakukan perhitungan dengan 4 keluaran, yaitu: recall, precision, acuraccy dan error rate.
\par
\item Pembacaan Confusion Matrix
\begin{enumerate}
\item Apabila hasil prediksi negatif dan data sebenarnya merupakan negatif.
\item Apabila hasil prediksi positif sedangkan nilai sebenarnya merupakan negatif.
\item Apabila hasil prediksi negatif sedangkan nilai sebenarnya merupakan positif.
\item Apabila hasil prediksi positif dan nilai sebenarnya merupakan positif.
\end{enumerate}
\par
\par
\item Pembuatan Confusion Matrix
\begin{enumerate}
\item Menentukan 4 proses klasifikasi yang akan digunakan dalam confusion matrix.
\item 4 Istilah tersebut ada True Positive ( TP ), True Negative ( TN ), False Positive ( FP ) dan False Negative ( FN ).
\item Kelompokkan klasifikasi tersebut bisa menggunakan klasifikasi biner
\item Akan menghasilkan keluaran berupa 2 Kelas ( Positif dan Negatif ) dan penentuan TP, FP ( 1 klasifikasi positif ) , FN dan TN ( 1 klasifikasi negatif ).
\item Contoh dasarnya nampak seperti langkah diatas
\item Istilahnya daat didefinisikan dengan objek lain namu dengan alur yang sama ( sesuai rumus baik klasifikasi dll ).
\end{enumerate}
\par

\item Ilustrasi Gambar
\par

\begin{figure}[ht]
\centering
\includegraphics[scale=0.6]{figures/confusion.jpg}
\caption{confusion matrix}
\label{gambar6}
\end{figure}
\end{itemize}

\par
\item Cara Kerja K-Fold Classification Dan Ilustrasi Gambar
\begin{enumerate}
\item Pertama-tama untuk total instance dibagi menjadi N bagian.
\item Fold ke-1 ( atau pertama ) adalah ketika bagian ke-1 menjadi data uji (testing data) dan sisanya menjadi data latih (training data).
\item Hitung akurasi ( berdasarkan porsi data tersebut. Persamaanya sebagai berikut :
\par (sigma) data klasifikasi
\par (sigma) total data uji
\par x 100 persen 
\item Fold ke-2 ( kedua ) adalah ketika bagian ke-2 menjadi data uji (testing data) dan sisanya menjadi data latih (training data). 
\item Kemudian dihitunglah akurasi berdasarkan porsi data yang telah ditentukan
\item Demikian seterusnya hingga mencapai fold ke-K. Hitung rata-rata akurasi dari K buah akurasi di atas. Rata-rata akurasi ini menjadi akurasi final atau akhir.
\end{enumerate}
\par
\begin{itemize}
\item Ilustrasi Gambar
\par

\begin{figure}[ht]
\centering
\includegraphics[scale=0.4]{figures/hasilk1.jpg}
\caption{k-fold classification 1}
\label{gambar7}
\end{figure}

\par
\item Decision Tree Dan Ilustrasi Gambar
\begin{itemize}
\item Pengertian Decision Tree
\par Decision Tree (Pohon Keputusan) adalah pohon dimana setiap cabangnyamenunjukkan pilihan diantara sejumlah alternatif pilihan yang ada, dan setiapdaunnya menunjukkan keputusan yang dipilih.Decision tree biasa digunakan untuk mendapatkan informasi untuk tujuanpengambilan sebuah keputusan. Decision tree dimulai dengan sebuah root node(titik awal) yang dipakai oleh user untuk mengambil tindakan. Dari node root ini,user memecahnya sesuai dengan algoritma decision tree. Hasil akhirnya adalahsebuah decision tree dengan setiap cabangnya menunjukkan kemungkinansekenario dari keputusan yang diambil serta hasilnya.
\par

\end{itemize}
\par

\par
\begin{itemize}
\item Ilustrasi Gambar
\par


\begin{figure}[ht]
\centering
\includegraphics[scale=0.5]{figures/decisiontree.jpg}
\caption{decision tree}
\label{gambar8}
\end{figure}

\par
\end{itemize}
\item Information Gain Dan Entropi
\begin{itemize}
\item Pengertian Information Gain
\par Information gain adalah salah satu atribute selection measure yang digunakan untuk memilih test atribute tiap node pada tree. Atribut dengan information gain tertinggi dipilih sebagai test atribut dari suatu node. Ada 2 kasus berbeda pada saat penghitungan Information Gain, pertama untuk kasus penghitungan atribut tanpa missing value dan kedua, penghitungan atribut dengan missing value.
\par

\end{itemize}
\item Ilustrasi Gambar
\par


\begin{figure}[ht]
\centering
\includegraphics[scale=0.6]{figures/information1.jpg}
\caption{informaion gain 1}
\label{gambar9}
\end{figure}


\par
\item Pengertian Entropi
\par Adalah suatu parameter untuk mengukur tingkat keberagaman (heterogenitas) dari kumpulan data. Semakin heterogen, nilai entropi semakin besar. 

\par

\begin{figure}[ht]
\centering
\includegraphics[scale=0.5]{figures/entropi.jpg}
\caption{entropi}
\label{gambar10}
\end{figure}

\par
\end{itemize}

\end{enumerate}

