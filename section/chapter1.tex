\chapter{Mengenal Kecerdasan Buatan dan Scikit-Learn}
Buku umum yang digunakan adalah \cite{russell2016artificial} dan  
untuk sebelum UTS menggunakan buku \textit{Python Artificial Intelligence Projects for Beginners}\cite{eckroth2018python}.
Dengan praktek menggunakan python 3 dan editor anaconda dan library python scikit-learn.
Tujuan pembelajaran pada pertemuan pertama antara lain:
\begin{enumerate}
\item
Mengerti definisi kecerdasan buatan, sejarah kecerdasan buatan, perkembangan dan penggunaan di perusahaan
\item
Memahami cara instalasi dan pemakaian sci-kit learn
\item
Memahami cara penggunaan variabel explorer di spyder
\end{enumerate}
Tugas dengan cara dikumpulkan dengan pull request ke github dengan menggunakan latex pada repo yang dibuat oleh asisten riset.

\section{Teori}
Praktek teori penunjang yang dikerjakan :
\begin{enumerate}
\item
Buat Resume Definisi, Sejarah dan perkembangan Kecerdasan Buatan, dengan bahasa yang mudah dipahami dan dimengerti. Buatan sendiri bebas plagiat[hari ke 1](10)
\item
Buat Resume mengenai definisi supervised learning, klasifikasi, regresi dan unsupervised learning. Data set, training set dan testing set.[hari ke 1](10)
\end{enumerate}

\section{Instalasi}
Membuka https://scikit-learn.org/stable/tutorial/basic/tutorial.html. Dengan menggunakan bahasa yang mudah dimengerti dan bebas plagiat. 
Dan wajib skrinsut dari komputer sendiri.
\begin{enumerate}
\item
Instalasi library scikit dari anaconda, mencoba kompilasi dan uji coba ambil contoh kode dan lihat variabel explorer[hari ke 1](10)
\item
Mencoba Loading an example dataset, menjelaskan maksud dari tulisan tersebut dan mengartikan per baris[hari ke 1](10)
\item
Mencoba Learning and predicting, menjelaskan maksud dari tulisan tersebut dan mengartikan per baris[hari ke 2](10)
\item
mencoba Model persistence, menjelaskan maksud dari tulisan tersebut dan mengartikan per baris[hari ke 2](10)
\item 
Mencoba Conventions, menjelaskan maksud dari tulisan tersebut dan mengartikan per baris[hari ke 2](10)
\end{enumerate}


\section{Penanganan Error}
Dari percobaan yang dilakukan di atas, apabila mendapatkan error maka:

\begin{enumerate}
	\item
	skrinsut error[hari ke 2](10)
	\item
Tuliskan kode eror dan jenis errornya [hari ke 2](10)
	\item
Solusi pemecahan masalah error tersebut[hari ke 2](10)

\end{enumerate}

\section{Fadila/1164072}
\subsection{Teori}
Teori mencakup resume dari beberapa pembahasan. yaitu :
\begin{enumerate}
\item Tentang Kecerdasan Buatan
\begin{itemize}
\item Definisi Kecerdasan Buatan.
\par Kecerdasan Buatan biasa disebut dengan istilah AI ( Artificial Intelligence ) . AI sendiri merupakan suatu cabang dalam bidang sains komputer sains dimana mengkaji tentang bagaimana cara untuk melengkapi sebuah komputer dengan kemampuan atau kepintaran layaknya atau mirip dengan yang dimiliki manusia. Sebagai contoh, sebagaimana komputer dapat berkomunikasi dengan pengguna baik menggunakan kata, suara maupun lain sebagainya . Dengan kemampuan ini, diharapkan komputer mampu mengambil keputusan sendiri untuk berbagai kasus yang ditemuinya kemudian itulah yang disebut dengan kecerdasan buatan.
\par  Kecerdasan buatan makin canggih dengan kemampuan komputer dalam memperbarui pengetahuannya dengan banyaknya testing dan perkembangan target analisa. Untuk kecerdasan buatan ada banyak contoh dan jenisnya. Salah satu contoh yang paling terkenal dari Artificial Intelligence ialah Google Assistant. Google Assistant digunakan untuk kemudahan user dalam menemukan berbagai hal maupun penyettingan langsung terhadap smartphone yang digunakan dan masih banyak lagi.
\item Sejarah Kecerdasan Buatan
\par Artificial intelligence merupakan inovasi baru di bidang ilmu pengetahuan. Mulai terbentuk sejak adanya komputer modern dan kira-kira terjadi sekitaran tahun 1940 dan 1950. Ilmu pengetahuan komputer ini khusus ditujukan dalam perancangan otomatisasi tingkah laku cerdas dalam sistem kecerdasan komputer. 
\par Pada awalnya, kecerdasan buatan hanya ada di universitas-universitas dan laboratorium penelitian, serta hanya sedikit produk yang dihasilkan dan dikembangkan. Menjelang akhir 1970-an dan 1980-an, mulai dikembangkan secara penuh dan hasilnya berangsur-angsur dipublikasikan di khalayak umum. 
\par

\begin{figure}[ht]
\centering
\includegraphics[scale=0.5]{figures/contoh.jpg}
\caption{capturing}
\label{contoh}
\end{figure}

\par Jika kita berbicara tentang AI atau Artificial Intelligence maka kita tidak bisa melupakan seorang sosok yang sangat terkenal pada bidang tersebut yaitu bapak John McCarthy. 
McCarthy mendapatkan gelar sarjana matematika dari California Institute of Technology (Caltech) pada September 1948. Dari masa kuliahnya itulah ia mulai mengembangkan ketertarikannya pada mesin yang dapat menirukan cara berpikir manusia. McCarthy kemudian melanjutkan pendidikan ke program doktoral di Princeton University.

\par McCarthy kemudian mendirikan dua lembaga penelitian kecerdasan buatan. Kedua lembaga AI itu adalah Stanford Artificial Intelligence Laboratory dan MIT Artificial Inteligence Laboratory. Di lembaga-lembaga inilah bermunculan inovasi pengembangan AI yang meliputi bidang human skill, vision, listening, reasoning dan movement of limbs. Bahkan Salah satu lembaga yang didirikan itu, Stanford Artificial Intelligence pernah mendapat bantuan dana dari Pentagon untuk membuat teknologi-teknologi luar angkasa.

\item Perkembangan Kecerdasan Buatan
\par Teknologi Artificial Intelligence semakin ramai dibahas dalam berbagai diskusi teknologi di seluruh dunia.Menurut kebanyakan orang, pekerjaan seperti kasir, operator telepon, pengendara truk, dan lainnya sangat berpeluang besar untuk tergantikan oleh Artificial Intelligence. Mengapa terjadi hal demikian? dikarenakan memang bahwa AI lebih ungul dalam hal kinerja, fitur dan lain sebagainya. Namun, dalam beberapa aspek memang pekerja manusia masih unggul dibandingkan AI itu sendiri.
\par Para generasi muda yang ada di dunia terutama di daerah Asia terlihat sudah memahami fungsi dan efek dari AI dalam kehidupan kita sehari-hari. Berdasarkan survei yang dilakukan oleh Microsoft, terdapat 39 persen responden yang mempertimbangkan untuk menggunakan mobil tanpa pengemudi dan 36 persen lainnya setuju bahwa robot masa depan dengan software untuk beroperasi mampu meningkatkan produktivitas. Dari survey tersebut kita sebagai pengguna AI harus lebih bijaksana dalam pengembangan dan penggunaan dari AI sehingga tanpa memberikan efek samping terhadap etos kerja dan keseharian kita sebagai pengguna dalam kehidupan sehari-hari.
\end{itemize}
\item Tentang Pengertian Terhadap Ilmu Yang Lain
\begin{itemize}
\item Supervised Learning adalah pendekatan dimana sudah terdapat data yang dilatih selain itu juga terdapat variable yang ditargetkan sehingga tujuan dari pendekatan ini yaitu mengkelompokan suatu data ke data yang sudah ada.
\par
\item Klasifikasi adalah pembagian sesuatu menurut kelas-kelas ( class ). Menurut Ilmu Pengetahuan, Klasifikasi merupakan proses pengelompokkan benda berdasarkan ciri-ciri persamaan dan juga perbedaan.
\par
\item Regresi adalah metode analisis statistik yang digunakan untuk melihat pengaruh antara dua ataupun lebih variabel.
\par
\item Unsupervised Learning berbeda dengan Supervised Leraning. Perbedaannya ialah unsupervised learning tidak memiliki data latih, sehingga dari data yang ada kita mengelompokan data tersebut menjadi 2  ataupun 3 bagian dan seterusnya.
\par
\item Dataset adalah objek yang merepresentasikan data dan juga relasi yang ada di memory. Strukturnya mirip dengan data di database, namun bedanya dataset berisi koleksi dari data table dan data relation. 
\par
\item Training Set adalah set digunakan oleh algoritma klassifikasi . Dapat dicontohkan dengan :  decision tree, bayesian, neural network dll. Semuanya dapat digunakan untuk membentuk sebuah model classifier. 
\par
\item Testing Set adalah set yang digunakan untuk mengukur sejauh mana classifier berhasil melakukan klasifikasi dengan benar. 
\par
\end{itemize}
\end{enumerate} 


\subsection{Instalasi}
Untuk Instalasinya mencakup i beberapa pembahasan dan tutorial. yaitu :
\begin{enumerate}
\item Instalasi Scikit-Learn Dari Anaconda 
\begin{itemize}
\item Instalasi Anaconda
\begin{enumerate}
\item Pertama-tama silahkan pastikan bahwa anda telah melakukan instalasi software Anaconda.
\item Apabila belum, silahkan buka web browser anda untuk melakukan pengunduhan software Anaconda
\item Setelah terunduh, silahkan klik kanan lalu run administrator pada software Anaconda
\item Silahkan lakukan penginstalan dengan menekan tombol install pada tampilan instalasi
\item Kemudian tekan tombol next maka akan sampai pada tampilan diatas

\par

\begin{figure}[ht]
\centering
\includegraphics[scale=0.3]{figures/ana1.jpg}
\caption{install anaconda 1}
\label{contoh}
\end{figure}

\par
\item Selanjutnya apabila instalan tersebut telah selesai maka silahkan menekan tombol next
\item Tampilan selanjutnya akan seperti ini

\par

\begin{figure}[ht]
\centering
\includegraphics[scale=0.3]{figures/ana2.jpg}
\caption{install anaconda 2}
\label{contoh}
\end{figure}

\par

\item Apabila tampilannya telah sesuai dengan contoh gambar maka instalasi telah selesai
\end{enumerate}
\end{itemize}

\begin{itemize}
\item Instalasi Library Scikit Learn
\begin{enumerate}
\item Silahkan membuka web browser untuk melakukan pengunduhan untuk library scikit dari anaconda.
\item Silahkan mengunjungi halaman ini untuk melakukan pengunduhan library scikit dari anaconda.
\par https://anaconda.org/anaconda/scikit-learn.
\item Setelah terdownload silahkan melakukan instalasi lanjutan menggunakan Command Prompt
\item Silahkan masukkan perintah berikut untuk melakukan pengecekan bahwa anaconda anda telah terpasang dengan baik.
\par conda --version
\par python --version
\item Tampilannya akan nampak seperti berikut :
\par

\begin{figure}[ht]
\centering
\includegraphics[scale=0.3]{figures/scikit1.jpg}
\caption{Pengecekan Anaconda}
\label{contoh}
\end{figure}

\par
\item Selanjutnya silahkan masukkan perintah berikut untuk melakukan instalasi pip sckit-learn
\par perintahnya : pip install -U scikit-learn
\item Tampilannya akan nampak seperti berikut :
\par

\begin{figure}[ht]
\centering
\includegraphics[scale=0.2]{figures/scikit3.jpg}
\caption{instalasi pip scikit-learn}
\label{contoh}
\end{figure}

\par
\item Selanjutnya silahkan masukkan perintah berikut untuk melakukan instalasi conda sckit-learn
\par perintahnya : conda install scikit-learn
\item Tampilannya akan nampak seperti berikut :
\par

\begin{figure}[ht]
\centering
\includegraphics[scale=0.2]{figures/scikit4.jpg}
\caption{instalasi conda scikit-learn}
\label{contoh}
\end{figure}

\par
\par
\par
\item Apabila telah dipraktekan seperti langkah-langkah dan menghasilkan tampilan seperti contoh diatas, maka instalasi scikit-learn dari anaconda berhasil dilakukan
\par
\item Kemudian untuk pengujian yang lain yaitu pengujian untuk mengecek codingan anaconda
\par
\item Contoh uji coba codingannya dapat dilihat pada gambar berikut
\par
\begin{figure}[ht]
\centering
\includegraphics[scale=0.5]{figures/3.jpg}
\caption{uji coba codingan}
\label{contoh}
\end{figure}
\par
\item Berdasarkan pengujian tersebut maka dapat dipastikan bahwa anaconda telah ter-include ke dalam python dan dieksekusi dengan script python
\item Setelah pengeksekusiannya berdasarkan scripts python, terdapatlah keluaran yang sesuai 
\item Keluaran tersebut yang menandakan bahwa anacondanya berfungsi dengan baik.
\end{enumerate}
\end{itemize}


\par
\item Loading An Example Dataset
\begin{itemize}
\item Penerapan Loading An Example Dataset Pada Python Di CMD
\begin{enumerate}
\item Pertama-tama silahkan buka command prompt di laptop anda
\item Selanjutnya masuk ke python 
\item Setelah masuk kedalam python, silahkan masukkan perintah seperti pada gambar.

\begin{figure}[ht]
\centering
\includegraphics[scale=0.5]{figures/4.jpg}
\caption{pengujian loading an example dataset}
\label{contoh}
\end{figure}

\par 
\item Apabila tampilanya telah nampak seperti gambar berikut , maka pengujiannya telah selesai dan berhasil.
\end{enumerate}

\par
\item Penjelasan Perintah Yang Di Uji
\begin{enumerate}
\item Perhatikan perintah yang telah dieksekusi ini :

\par
\begin{figure}[ht]
\centering
\includegraphics[scale=0.7]{figures/ok4.jpg}
\caption{pengujian loading an example dataset}
\label{contoh}
\end{figure}
\par

\item Penjelasan untuk baris pertama ialah : 
\par Perintahnya yaitu memasukkan dan memanggil dataset dari sklearn
\par
\par
\item Penjelasan untuk baris kedua ialah :
\par Terdapat variabel baru yaitu iris. Dimana variabel iris memanggil datasets dan di dalamnya akan ngeload ( menampilkan ) load iris.
\par
\item Penjelasan untuk baris ketiga ialah :
\par Kemudian ada juga variabel baru lainnya yaitu digits yang akan memanggil dataset dan di dalamnya akan ngeload ( menampilkan ) load digits
\par
\item Selanjutnya untuk perintah Print( digits.data ) ditujukan untuk me-
\par nampilkan output dari pengeksekusian variabel digits dan akan berupa data.
\par
\item Hasilnya printnya sebagai berikut :
\par

\begin{figure}[ht]
\centering
\includegraphics[scale=0.7]{figures/okk4.jpg}
\caption{hasil print uji cobat}
\label{contoh}
\end{figure}

\par
\item Untuk penjelasan uji cobanya sudah selesai.
\end{enumerate}
\end{itemize}

\par
\par
\item Learning And Predicting
\begin{itemize}
\item Penerapan Learning Dan Predicting Pada Python Di CMD
\begin{enumerate}
\item Pertama-tama silahkan buka command prompt di laptop anda
\item Selanjutnya masuk ke python 
\item Setelah masuk kedalam python, silahkan masukkan perintah ( scriptsnya ) sesuai dengan contoh yang akan diberikan

\begin{figure}[ht]
\centering
\includegraphics[scale=0.3]{figures/1.jpg}
\caption{pengujian learning dan predicting}
\label{contoh}
\end{figure}

\par
\item Contohnya nampak sepeti pada gambar.
\item Apabila tampilanya telah nampak seperti gambar diatas, maka pengujiannya telah selesai dan berhasil.
\end{enumerate}

\par
\item Penjelasan Perintah Yang Di Uji, ( sesuai dengan contoh perintah pada gambar).
\begin{enumerate}
\item Penjelasan untuk baris 1 ialah : 
\par Memanggil dan memasukkan datasets dari sklearn
\par 
\par
\item Penjelasan untuk baris 2 ialah :
\par membuat variabel iris yang memanggil load data pada datasets tanpa parameter
\par
\item Penjelasan untuk baris 3 ialah :
\par membuat variabel digits yang memanggil load digits dari datasets tanpa parameter
\par
\item Penjelasan untuk baris 4 ialah :
\par melakukan perintah print data yang akan menampilkan data dari eksekusi variabel digits
\par
\item Penjelasan untuk baris 5 ialah :
\par hasil eksekusi
\par
\item Penjelasan untuk baris 6 ialah :
\par membuat clf pada module fit metode dengan menggunakan 2 paramater yaitu digits data dan digits target
\par
\item Penjelasan untuk baris 7 ialah :
\par svc ini mengimplementasikan yang namanya data berupa klasifikasi dukungan vektor.
\par
\item Untuk penjelasan uji cobanya sudah selesai.
\par
\end{enumerate}
\end{itemize}


\par
\item Model Persistence
\begin{itemize}
\item Penerapan Model Persistence Pada Python Di CMD
\begin{enumerate}
\item Pertama-tama silahkan buka command prompt di laptop anda
\item Selanjutnya masuk ke python 
\item Setelah masuk kedalam python, silahkan masukkan perintah ( scriptsnya ) sesuai dengan contoh yang akan diberikan

\begin{figure}[ht]
\centering
\includegraphics[scale=0.3]{figures/modelpre1.jpg}
\caption{pengujian model persistence 1 }
\label{contoh}
\end{figure}

\par

\par

\begin{figure}[ht]
\centering
\includegraphics[scale=0.4]{figures/modelpre2.jpg}
\caption{pengujian model persistence 2 }
\label{contoh}
\end{figure}

\par
\item Contohnya nampak sepeti pada gambar.
\item Apabila tampilanya telah nampak seperti gambar diatas, maka pengujiannya telah selesai dan berhasil.
\end{enumerate}

\par
\item Penjelasan Perintah Yang Di Uji, ( sesuai dengan contoh perintah pada gambar).
\item Model Persistence 1
\begin{enumerate}
\item Penjelasan untuk baris 1 ialah : 
\par Memanggil ataupun memasukkan datasets dari sklear.
\par
\par
\item Penjelasan untuk baris 2 ialah :
\par Memanggil ataupun memasukkan svm dari sklearn
\par
\item Penjelasan untuk baris 3 ialah :
\par Membuat variabel baru yaitu clf dimana akan memanggil svm.SVC yang telah mendefinisikan sebuah parameter yaitu gamma.
\par
\item Penjelasan untuk baris 4 ialah :
\par Membuat variabel baru lainnya yaitu iris dimana akan memanggil datasets yang didalamnya akan ngeload data iris.
\par
\item Penjelasan untuk baris 5 ialah :
\par Membuat variabel baru lainnya untuk X dan Y dengan mendefinisikan pemanggilan iris data dan iris target.
\par
\item Penjelasan untuk baris 6 ialah :
\par Variabel clf dipasang pada model fit metode dengan parameter X dan Y
\par
\item Penjelasan untuk baris 7 ialah : 
\par Kemudian svc ini mengimplementasikan yang namanya data berupa klasifikasi dukungan vektor.
\par
\par
\end{enumerate}
\item Model Persistence 2
\begin{enumerate}
\item Penjelasan untuk baris 1  ialah :
\par Melakukan pemanggilan terhadap library pickle
\par
\item Penjelasan untuk baris 2 ialah :
\par Membuat variabel baru yaitu s dengan pemanggilan pickle dumps dengan pendefinisian variabel clf
\par
\item Penjelasan untuk baris 3  ialah :
\par Membuat variabel clf2 dengan memanggil pickle loads dengan pendefinisian variabel s
\par
\item Penjelasan untuk baris 4  ialah :
\par Prediksi nilai baru dari variabel clf2 dengan parameternya yaitu X
\par
\item Penjelasan untuk baris 5  ialah :
\par set array dari prediksi variabel clf2
\par
\item Penjelasan untuk baris 6 ialah : 
\par Parameter X dengan array 0 akan menghasilkan set array 0 juga
\par
\par
\item Untuk penjelasan uji cobanya sudah selesai.
\par
\end{enumerate}
\end{itemize}


\par
\item Conventions
\begin{itemize}
\item Penerapan Conventions Pada Python Di CMD
\begin{enumerate}
\item Type Casting
\begin{itemize}
\item Pertama-tama silahkan buka command prompt di laptop anda
\item Selanjutnya masuk ke python 
\item Setelah masuk kedalam python, silahkan masukkan perintah ( scriptsnya ) sesuai dengan contoh yang akan diberikan
\par

\begin{figure}[ht]
\centering
\includegraphics[scale=0.3]{figures/typecast1.jpg}
\caption{pengujian type casting 1}
\label{contoh}
\end{figure}

\par

\par
\par

\begin{figure}[ht]
\centering
\includegraphics[scale=0.3]{figures/typecast2.jpg}
\caption{pengujian type casting 2}
\label{contoh}
\end{figure}

\par
\par
\end{itemize}

\item Refting And Updating Parameters
\begin{itemize}
\item Pertama-tama silahkan buka command prompt di laptop anda
\item Selanjutnya masuk ke python 
\item Setelah masuk kedalam python, silahkan masukkan perintah ( scriptsnya ) sesuai dengan contoh yang akan diberikan
\par

\begin{figure}[ht]
\centering
\includegraphics[scale=0.4]{figures/reftingupdating.jpg}
\caption{pengujian refting and updating }
\label{contoh}
\end{figure}

\par
\end{itemize}
\par
\item Multiclass And Multilable Fitting
\begin{itemize}
\item Pertama-tama silahkan buka command prompt di laptop anda
\item Selanjutnya masuk ke python 
\item Setelah masuk kedalam python, silahkan masukkan perintah ( scriptsnya ) sesuai dengan contoh yang akan diberikan
\par

\begin{figure}[ht]
\centering
\includegraphics[scale=0.4]{figures/multiclass1.jpg}
\caption{pengujian multiclass and multiable 1 }
\label{contoh}
\end{figure}

\par
\par

\begin{figure}[ht]
\centering
\includegraphics[scale=0.5]{figures/multiclass2.jpg}
\caption{pengujian multiclass and multiable 2 }
\label{contoh}
\end{figure}

\par
\end{itemize}

\item Apabila semua proses yang telah dilakukan terlihat seperti contoh-contoh diatas, maka pengujian telah selesai.
\end{enumerate}

\par
\item Penjelasan Perintah Yang Di Uji, berdasarkan contoh perintah-perintah diatas :
\begin{enumerate}
\item Type Casting :
\begin{itemize}
\item Type Casting 1
\par
\begin{itemize}
\item Penjelasan untuk baris 1 ialah : 
\par Memasukkan dan memanggil module / library numphy sebagai np
\par
\par
\item Penjelasan untuk baris 2 ialah :
\par Memasukkan dan memanggil random projection dari sklearn
\par
\item Penjelasan untuk baris 3  ialah :
\par Membuat variabel rng, dimana memanggil np yang akan mengambil dan mengeksekusi random state dengan parameter 0
\par
\item Penjelasan untuk baris 4  ialah :
\par Membuat variabel baru lainnya yaitu X dengan memanggil variabel rng dengan 2 parameter yaitu 10 dan 2000
\par
\item Penjelasan untuk baris 5  ialah :
\par Membuat variabel X lagi namun dengan pemanggilan yang berbeda yaitu array dari np dengan parameternya x dan dtype='float32'.
\par
\item Penjelasan untuk baris 6 ialah :
\par Pemanggilan dtype dari variabel X
\par
\item Penjelasan untuk baris 7 ialah : 
\par Hasil dari pemanggilan dtype dari variabel X
\par
\par
\item Penjelasan untuk baris 8 ialah :
\par Membuat variabel transformer dengan pemanggilan gaussian random projection
\par
\item Penjelasan untuk baris 9 ialah :
\par Membuat variabel baru yaitu X new dengan memanggil variabel transformer yang berada pada model fit metode dengan parameternya yaitu X
\par
\item Penjelasan untuk baris 10 ialah :
\par Pemanggilan dtype dari variabel X new
\par
\item Penjelasan untuk baris 11 ialah :
\par Hasil dari pemanggilan dtype variabel X new
\par
\item Untuk penjelasan uji cobanya sudah selesai.
\par
\end{itemize}
\end{itemize}
\par
\par
\begin{itemize}
\item Type Casting 2
\par
\begin{itemize}
\item Penjelasan untuk baris 1 ialah : 
\par Memasukkan dan memanggil dataset dari sklearn
\par
\par
\item Penjelasan untuk baris 2 ialah :
\par Memasukkan dan memanggil scv dari sklearn
\par
\item Penjelasan untuk baris 3  ialah :
\par Membuat variabel iris dengan memanggil load iris dari datasets
\par
\item Penjelasan untuk baris 4  ialah :
\par Membuat variabel clf dengan memanggil svc dengan parameter gamma
\par
\item Penjelasan untuk baris 5  ialah :
\par Membuat clf pada module fit metode dengan 2 parameter yaitu iris data dan iris target
\par
\item Penjelasan untuk baris 6 ialah :
\par membuat variabel list dengan prediksi clf
\par
\item Penjelasan untuk baris 7 ialah : 
\par SVC mengimplementasikan yang namanya data berupa klasifikasi dukungan vektor.
\par
\par
\item Penjelasan untuk baris 8 ialah :
\par membuat variabel list dengan prediksi iris data 
\par
\item Penjelasan untuk baris 9 ialah :
\par hasil dari prediksi list clf
\par
\item Untuk penjelasan uji cobanya sudah selesai.
\par
\end{itemize}
\end{itemize}

\par
\par
\par
\par
\par
\item Refting And Updating Parameters :
\par
\begin{itemize}
\item Penjelasan untuk baris 1 ialah : 
\par Memasukkan dan memanggil module / library numphy sebagai np
\par
\par
\item Penjelasan untuk baris 2 ialah :
\par Memasukkan dan memanggil SVC dari sklearn.svm
\par
\item Penjelasan untuk baris 3  ialah :
\par Membuat variabel rng, dimana memanggil np yang akan mengambil dan mengeksekusi random state dengan parameter 0
\par
\item Penjelasan untuk baris 4  ialah :
\par Membuat variabel baru lainnya yaitu X dengan memanggil variabel rng dengan 2 parameter yaitu 100 dan 10
\par
\item Penjelasan untuk baris 5  ialah :
\par Membuat variabel Y dimana memanggil binomal dari rng dengan 3 parameter yaitu 1, 0.5 dan 100.
\par
\item Penjelasan untuk baris 6 ialah :
\par Memuat variabel baru lainnya itu X test dimana memanggil rand dari rng dengan 2 parameter yaitu 5 dan 10.
\par
\item Penjelasan untuk baris 7 ialah : 
\par Membuat variabel clf dengan mendefinisikan SVC tanpa parameter
\par
\par
\item Penjelasan untuk baris 8 ialah :
\par Melakukan parameter set dari clf dengan parameter kernel='linear'.
\par
\item Penjelasan untuk baris 9 ialah :
\par SVC mengimplementasikan yang namanya data berupa klasifikasi dukungan vektor.
\par
\item Penjelasan untuk baris 10 ialah :
\par Membuat prediksi clf dengan parameternya yaitu variabel X test
\par
\item Penjelasan untuk baris 11 ialah :
\par set array yang dihasilkan oleh prediksi clf
\par
\item Penjelasan untuk baris 12 ialah :
\par Melakukan parameter set dari clf dengan parameter kernel='rbf', gamma='scale' dan fit yaitu X dan Y
\par
\item Penjelasan untuk baris 13 ialah :
\par  SVC mengimplementasikan yang namanya data berupa klasifikasi dukungan vektor.
\par
\item Penjelasan untuk baris 14 ialah :
\par Membuat prediksi clf dengan parameternya yaitu variabel X test
\par
\item Penjelasan untuk baris 15 ialah :
\par set array yang dihasilkan oleh prediksi clf
\par
\item Untuk penjelasan uji cobanya sudah selesai.
\par

\end{itemize}



\par
\par
\par
\par
\par
\item Multiclass And Multilable Fitting
\par
\begin{itemize}
\item Multiclass And Multilable Fitting 1 
\begin{itemize}
\item Penjelasan untuk baris 1 ialah : 
\par Memasukkan dan memanggil SCV dari sklearn.svm
\par
\par
\item Penjelasan untuk baris 2 ialah :
\par Memasukkan dan memanggil OneVsRestClassifier dari sklearn.svm
\par
\item Penjelasan untuk baris 3  ialah :
\par Memasukkan dan memanggil LabelBinarizer dari sklearn.preprocessing
\par
\item Penjelasan untuk baris 4  ialah :
\par Membuat variabel X dengan beberapa parameter
\par
\item Penjelasan untuk baris 5  ialah :
\par Membuat variabel Y dengan beberapa parameter
\par
\item Penjelasan untuk baris 6 ialah :
\par Membuat variabel classif dengan memanggil OneVsRestClassifier yang didalamnya terdapat 2 parameter yaitu gamma dan random state.
\par
\item Penjelasan untuk baris 7 ialah : 
\par Membuat prediksi parameter X dari variabel classif yang berada dalam module fit metode dengan parameter X dan Y
\par
\par
\item Penjelasan untuk baris 8 ialah :
\par set array dari prekdiksi calssif
\par
\item Penjelasan untuk baris 9 ialah :
\par Membuat variabel Y baru dengan memanggil LabeBinazier tanpa parameter dan fit transform dengan parameter Y
\par
\item Penjelasan untuk baris 10 ialah :
\par Membuat prediksi parameter X dari variabel classif yang berada dalam module fit metode dengan parameter X dan Y
\par
\item Penjelasan untuk baris 11 ialah :
\par set array dari prekdiksi calssif
\par
\end{itemize}
\end{itemize}
\par
\par
\begin{itemize}
\item Multiclass And Multilable Fitting 2 
\par
\begin{itemize}
\item Penjelasan untuk baris 1 ialah : 
\par Memasukkan dan memanggil MultiLabelBinarizer dari sklearn.processing
\par
\par
\item Penjelasan untuk baris 2 ialah :
\par Membuat variabel Y dengan beberapa parameter / nilai
\par
\item Penjelasan untuk baris 3  ialah :
\par Membuat variabel Y dengan memanggil MultiLLabeLBinarizer tanpa parameter dan Fit transform dengan parameter y
\par
\item Penjelasan untuk baris 4  ialah :
\par Membuat prediksi dengan parameter X dari classif pada module fit metode dengan parameter X dan Y
\par
\item Penjelasan untuk baris 5  ialah :
\par set array dari prediksi classif
\par
\item Untuk penjelasan uji cobanya sudah selesai.
\par
\end{itemize}
\end{itemize}
\end{enumerate}
\end{itemize}
\end{enumerate}




\subsection{Penanganan Error}
Terdapat beberapa error pada pengujian diatas dan penanganannya, yaitu:
\begin{enumerate}
\item Model Persistence
\begin{itemize}
\item Errornya ditandai dengan tidak terdefinisinya module joblib pada komputer
\item Hal itulah yang menyebabkan tidak terprosesnya perintah terkait
\par

\begin{figure}[ht]
\centering
\includegraphics[scale=0.6]{figures/er.jpg}
\caption{error model persistence}
\label{contoh}
\end{figure}

\par
\par
\par
\end{itemize}
\item Penanganan Model Persistence
\begin{itemize}
\item Pertama-tama silahkan membuka command prompt
\item Kemudian masukkan perintah untuk melakukan instalasi module joblib
\par perintahnya ialah : pip install joblib
\item hasilnya akan nampak seperti pada gambar yang ditampilkan
\par

\begin{figure}[ht]
\centering
\includegraphics[scale=0.4]{figures/penanganan1.jpg}
\caption{penanganan error model persistence }
\label{contoh}
\end{figure}

\par
\end{itemize}
\end{enumerate}

\par
\begin{enumerate}
\item Pengujian Penanganan Model Persistence
\begin{itemize}
\item Setelah melakukan penginstalan maka kita harus menguji keberhasilan penginstalan
\item Caranya dengan mengecek lewat command prompt bahwa module joblibnya telah terdefinisikan di python
\item Silahkan ketikkan perintah python, lalu masukkan peritah sebagai berikut :
\par from joblib, import dump, load
\par
\item Maka hasilnya akan nampak seperti pada gambar yang ditampilkan

\begin{figure}[ht]
\centering
\includegraphics[scale=0.5]{figures/penanganan2.jpg}
\caption{penanganan error model persistence }
\label{contoh}
\end{figure}

\par
\end{itemize}
\end{enumerate}