\chapter{Experiment and Result}
brief of experiment and result.
\section{Experiment}
Please tell how the experiment conducted from method.

\section{Result}
Please provide the result of experiment

\section{Lusia Violita Aprilian/1164080}

\subsection{Teori}
\begin{enumerate}
\item Klasifikasi teks
	\par Klasifikasi Dokumen / Teks adalah salah satu tugas penting dan tipikal dalam supervised machine learning (ML). Menetapkan kategori pada dokumen, yang dapat berupa halaman web, buku perpustakaan, artikel media, galeri, dll. Memiliki banyak aplikasi seperti mis. penyaringan spam, perutean email, analisis sentimen dll. 
	\begin{figure}[ht]
		\centering
		\includegraphics[scale=0.5]{figures/m1.jpg}
		\caption{Lusia-Klasifikasi teks}
		\label{contoh}
	\end{figure}
	
\item Klasifikasi Bunga tidak dapat penggunakan machine learning
	\par Klasifikasi bunga tidak dapat menggunakan machine learning karena memiliki masalah input yang serupa namun output yang berbeda atau 'noise'. Yang dimaksud dengan noise adalah contoh output yang direkam bukan seperti seharusnya. Misalnya saja kita secara implisit berasumsi bahwa contoh bunga kita telah diklasifikasikan dengan benar. Tetapi ini harus dilakukan dengan seseorang yang tepat, seperti seorang ahli botani. Seorang ahli botani ahli harus melihat contoh bunga dan berkata: " ini adalah setosa ... ini adalah virginica ", dan dengan demikian bertindak sebagai "guru" yang memungkinkan mesin untuk belajar. Tetapi bagaimana jika guru itu melakukan kesalahan? Selain itu, selalu ada peluang untuk memperkenalkan kesalahan saat merekam data. Noise juga ditemukan dalam pengukuran, yang selalu sedikit bermasalah karena alat dan sensor kami tidak sempurna dan hanya bekerja pada tingkat presisi tertentu.
	\begin{figure}[ht]
		\centering
		\includegraphics[scale=0.5]{figures/m2.jpg}
		\caption{Lusia-Klasifikasi bunga}
		\label{contoh}
	\end{figure}

\item Teknik pembelajaran mesin pada teks YouTube
	\par Dengan menggunakan kasus seperti rekomendasi video yang terdapat pada fiturnya, Machine Learning pada YouTube memperhatikan apa saja yang menarik perhatian para penggunanya. Ketika kita sedang menonton di YouTube, pada sebelah kanan terdapat 'Up Next' yang menampilkan beberapa video serupa yang sedang ditonton. Dan ketika mengklik salah satu video dari baris tersebut, maka YouTube akan mengingatnya dan menggunakan kata yang tertera sebagai referensi. 
	\begin{figure}[ht]
		\centering
		\includegraphics[scale=0.5]{figures/m3.jpg}
		\caption{Lusia-Teknik YouTube}
		\label{contoh}
	\end{figure}

\item Vectorisasi Data
	\begin{itemize}
		\item Maksud dari Vectorisasi Data merupakan Pemecahan dan Pembagian Data.
	\end{itemize}
	
\item Bag of word
	\par Bag-of-words adalah cara untuk merepresentasikan data teks saat memodelkan teks dengan algoritma pembelajaran mesin.
	\begin{figure}[ht]
		\centering
		\includegraphics[scale=0.5]{figures/m5.jpg}
		\caption{Lusia-Bag of Word}
		\label{contoh}
	\end{figure}
	
\item TF-IDF
	\par TF-IDF merupakan istilah frekuensi - frekuensi dokumen terbalik, adalah ukuran penilaian yang banyak digunakan dalam pengambilan informasi (IR) atau peringkasan. TF-IDF dimaksudkan untuk mencerminkan seberapa relevan suatu istilah dalam dokumen yang diberikan. Intuisi di baliknya adalah bahwa jika sebuah kata muncul beberapa kali dalam sebuah dokumen, kita harus meningkatkan relevansinya karena itu harus lebih bermakna daripada kata-kata lain yang muncul lebih sedikit kali (TF). Pada saat yang sama, jika sebuah kata muncul berkali-kali dalam suatu dokumen tetapi juga di sepanjang banyak dokumen lain, mungkin itu karena kata ini hanya kata yang sering; bukan karena itu relevan atau bermakna (IDF).
	\begin{figure}[ht]
		\centering
		\includegraphics[scale=0.5]{figures/m6.jpg}
		\caption{Lusia-TF IDF}
		\label{contoh}
	\end{figure}
\end{enumerate}

\subsection{Praktek}
\begin{enumerate}
\item Aplikasi menggunakan pandas
	\par Berikut adalah aplikasi yang dibuat menggunakan pandas :
	
		\begin{figure}[ht]
		\centering
		\includegraphics[scale=0.5]{figures/n1a.jpg}
		\caption{Lusia-Pandas}
		\label{contoh}
		\end{figure}
		
	\begin{enumerate}
	\item 1 = memanggil library pandas sebagai pd
	\item 2 = membuat varible mas untuk membaca foile csv
	\item 3 = membuat varible untuk membuat data frame
	\item 4 = membuat variable dummy untuk mengubah kategori menjadi integer
	\item 5 = untuk memunculkan data teratas
	\item 6 = untuk menjoinkan atau menggabungkan data frame dengan dummy
	\end{enumerate}
	
	\par Berikut adalah hasilnya :
	
		\begin{figure}[ht]
		\centering
		\includegraphics[scale=0.5]{figures/n1b.jpg}
		\caption{Lusia-Hasil Pandas}
		\label{contoh}
		\end{figure}	
	
\item Memecah data frame
	\par Berikut untuk memecah data frame menjadi dua :
	
		\begin{figure}[ht]
		\centering
		\includegraphics[scale=0.5]{figures/n2a.jpg}
		\caption{Lusia-Pecah data}
		\label{contoh}
		\end{figure}
	
	\begin{enumerate}
	\item 1 = split data training 0-450 data pertama
	\item 2 = split data test sisa dari 0-450 data pertama
	\end{enumerate}
	
	\par Berikut adalah hasilnya :
	
		\begin{figure}[ht]
		\centering
		\includegraphics[scale=0.5]{figures/n2b.jpg}
		\caption{Lusia-Hasil Pecah data}
		\label{contoh}
		\end{figure}
		
\item Vektorisasi dan klasifikasi Decission Tree Katty Perry
	\begin{itemize}
	\item Berikut adalah vektorisasi dan klasifikasi Katty Perry
		\begin{figure}[ht]
		\centering
		\includegraphics[scale=0.5]{figures/n3a.jpg}
		\caption{Lusia-Vektorisasi dan klasifikasi}
		\label{contoh}
		\end{figure}
	\par Maksud dari gambar vektorisasi dan klasifikasi Katty Perry adalah hasil dari impor dataset, lalu import countvectorizer dari sklearn. Modul sklearn feature extraction digunakan untuk mengekstrak fitur dalam format yang didukung oleh algoritma pembelajaran mesin dari kumpulan data yang terdiri dari format seperti teks dan gambar. Lalu membuat variabel Dan CountVectorizer mengimplementasikan tokenization dan penghitungan kejadian dalam satu kelas. lalu membuat variabel dvec untuk mempelajari dataset. Membuat variabel daptarkata yang berfungsi untuk pemetaan array dari indeks integer fitur ke nama fitur. 
	
	\item Berikut adalah Decission Tree Katty Perry
		\begin{figure}[ht]
		\centering
		\includegraphics[scale=0.5]{figures/n3b.jpg}
		\caption{Lusia-Decission Tree Katty Perry}
		\label{contoh}
		\end{figure}
	\par Dalam gambar Decission Tree dijelaskan bahwa library tree dari sklearn. Dan mendifinisikan variable untuk memanggil Decission Tree Classifisier yang kemudian dilakukan fit atau pengujian. Lalu menggunakan prediksi dengan fungsi predict untuk memprediksi test. Yang terakhir memunculkan akurasi prediksi yaitu 0,86.
	\end{itemize}

\item Klasifikasikan dari data vektorisasi dengan klasifikasi SVM
	\par Berikut adalah klasifikasikan dari data vektorisasi dengan klasifikasi SVM
		\begin{figure}[ht]
		\centering
		\includegraphics[scale=0.5]{figures/n4.jpg}
		\caption{Lusia-Hasil klasifikasi SVM}
		\label{contoh}
		\end{figure}
	\par Dalam gambar SVM dijelaskan bahwa mula-mula file svm diimpor dari sklearn, lalu melakukan fit dari d train att dan d train label atau disebut dengan pengujian. Selanjutnya variable didifinisikan variable untuk melakukan prediksi dataset dengan SVM. Dan yang terakhir muncul hasilnya yaitu 0,42.
	
\item Klasifikasikan dari data vektorisasi dengan klasifikasi Decission Tree
	\par Maksud dari gambar vektorisasi adalah hasil dari impor dataset
	\par Berikut adalah Decission Tree
		\begin{figure}[ht]
		\centering
		\includegraphics[scale=0.5]{figures/n3b.jpg}
		\caption{Lusia-Decission Tree}
		\label{contoh}
		\end{figure}
	\par Dalam gambar Decission Tree dijelaskan bahwa library tree dari sklearn. Dan mendifinisikan variable untuk memanggil Decission Tree Classifisier yang kemudian dilakukan fit atau pengujian. Lalu menggunakan prediksi dengan fungsi predict untuk memprediksi test. Yang terakhir memunculkan akurasi prediksi yaitu 0,86.

\item Plot confusion matrix
	\par Berikut adalah hasil dari ploting confusion matrix
		\begin{figure}[ht]
		\centering
		\includegraphics[scale=0.5]{figures/n6.jpg}
		\caption{Lusia-ploting confusion matrix}
		\label{contoh}
		\end{figure}
	\par Dari gambar dijelaskan bahwa library numpy di import sebagai np. Lalu 	Opsi set printoption untuk menentukan cara angka floating point, array dan objek NumPy lainnya ditampilkan. Selanjutnya matplotlib pyplot figure untuk membuat figur atau gambar baru. Selanjutnya confution matrix dinormalisasikan. Dan yang terakhir hasil ditampilkan.

\item Program cross validation
	\par Berikut adalah hasil dari program cross validation
		\begin{figure}[ht]
		\centering
		\includegraphics[scale=0.5]{figures/n7.jpg}
		\caption{Lusia-Program cross validation}
		\label{contoh}
		\end{figure}
		
	\par Maksud dari hasil gambar tersebut adalah untuk mendefinisikan dataset untuk 'menguji' model.
	
\item Program pengamatan komponen informasi
	\par Berikut adalah hasil dari program pengamatan komponen informasi
		\begin{figure}[ht]
		\centering
		\includegraphics[scale=0.5]{figures/n8.jpg}
		\caption{Lusia-Program pengamatan komponen informasi}
		\label{contoh}
		\end{figure}
		
	\par Maksud dari hasil gambar tersebut adalah diagram informasi dari dataset yang digunakan.

\end{enumerate}

\subsection{Penanganan Error}
\begin{enumerate}
	\item skrinsut error
		\begin{figure}[ht]
		\centering
		\includegraphics[scale=0.5]{figures/o1.jpg}
		\caption{Lusia-skrinsut error}
		\label{contoh}
		\end{figure}
	\item Tuliskan kode eror dan jenis errornya
		\begin{itemize}
		\item Kode error = KeyError: 'test preparation course'
		\item Jenis error = KeyError
		\end{itemize}
	\item Solusi pemecahan masalah error
		\par Solusinya adalah dengan memasukkan salah satu atribut dari dataset file csv yang digunakan.
\end{enumerate}


\section{Fadila-1164072}
\subsection{Teori}
Penjelasan Tugas Harian 7 ( No 1-6 )
\begin{enumerate}
\item Pengertian Klasifikasi Teks Dan Ilustrasi Gambar
\begin{itemize}
\item Pengertian Klasifikasi Teks
\par Klasifikasi teks adalah proses pemberian tag atau kategori ke teks sesuai dengan isinya. Teks dapat menjadi sumber informasi yang sangat kaya, tetapi mengekstraksi wawasan darinya bisa sulit dan memakan waktu karena sifatnya yang tidak terstruktur.
\par
\item Ilustrasi Gambar
\par Penjelasan : Berdasarkan pengertian diatas, ada beberapa contoh yang bisa diterapkan. Untuk salah satu contoh dari klasifikasi data sendiri dapat diliat pada gambar berikut \ref{text-fadila}.
\begin{figure}[!hbtp]
\centering
\includegraphics[scale=0.3]{figures/text-fadila.jpg}
\caption{text-fadila}
\label{text-fadila}
\end{figure}
\par
\end{itemize}
\par
\par
\item Mengapa Klasifikasi Bunga Tidak Bisa Menggunakan Machine Learning Dan Ilustrasi Gambar
\begin{itemize}
\item  Mengapa Klasifikasi Bunga Tidak Bisa Menggunakan Machine Learning
\par Penjelasan : Untuk klasifikasi bunga tidak dapat menggunakan machine learning dikarenakan memiliki masalah input yang sama namun keluarannya (output) yang berbeda, biasanya output atau error ini disebut dengan istilah 'noise'. Noise sendiri merupakan output yang disimpan / ditangkap maupun direkam bukan seperti seharusnya ( keluaran yang diiginkan ). 
\par Apabila diberikan contoh, maka contohnya yaitu kita berasumsi secara implisit bahwa klarifikasi bunga yang kita lakukan sudah tepat dan kita melakukannya seperti seorang ahli tanaman. Namun pada hasilnya masih saja terjadi kesalahan. Selain itu, selalu ada peluang untuk memperkenalkan kesalahan saat merekam ataupun menyimpan data, maka harus dilakukan penelitian yang lebih rinci sehingga tidak menimbulkan 'noise' itu sendiri.
\par
\item Ilustrasi Gambar
\par Penjelasan : Berdasarkan pengertian diatas, ada beberapa contoh yang bisa diterapkan. Untuk salah satu contoh dari klasifikasi bunga sendiri dapat diliat pada gambar berikut \ref{bunga-fadila}.
\begin{figure}[!hbtp]
\centering
\includegraphics[scale=0.3]{figures/bunga-fadila.jpg}
\caption{bunga-fadila}
\label{bunga-fadila}
\end{figure}
\par
\end{itemize}
\par
\par
\item Teknik Pembelajaran Mesin Pada Teks Pada Kata-Kata Yang Digunakan Di Youtube Dan Ilustrasi Gambar
\begin{itemize}
\item  Teknik Pembelajaran Mesin Pada Teks Pada Kata-Kata Yang Digunakan Youtube
\par Penjelasan : Kita ambil sebuah kasus yang semua orang telah ketahui dan juga pahami. Kasus tersebut yaitu perekomendasian video dari pencarian menggunakan "text / kata" di  Youtube. Pada saat menggunakan Youtube terdapat Machine Learning yang bekerja dan memproses perintah ataupun aktivitas tersebut, dimana akan memfilter secara otomatis video yang disesuaikan dengan "keyword" yang kita masukkan sehingga memberikan keluaran video dengan keyword yang benar. 
\par Adapula fitur yang di dapatkan ketika sedang menonton Youtube. Tampilan sebelah kanan terdapat pilihan 'Next' atapun 'Suggestion' yang menam-pilkan beberapa video serupa sesuai dengan yang dicari atau sedang ditonton. Ketika mengklik salah satu video dari baris tersebut, maka Youtube akan mengingat dan menggunakan kata yang tertera sebagai referensi kembali sehingga akan memberikan kemudahan pada pencarian yang lainnya, Dan disitulah mesin belajar sendiri dan menyimpan data secara berkala sehingga berkembang. 
\par
\item Ilustrasi Gambar
\par Penjelasan : Berdasarkan pengertian diatas, ada beberapa contoh yang bisa diterapkan. Untuk salah satu contoh dari Mesin Teks Youtube sendiri dapat diliat pada gambar berikut \ref{youtube-fadila}.
\begin{figure}[!hbtp]
\centering
\includegraphics[scale=0.25]{figures/youtube-fadila.jpg}
\caption{youtube-fadila}
\label{youtube-fadila}
\end{figure}
\par
\end{itemize}
\par
\par
\item Vektorisasi Data
\begin{itemize}
\item Maksud Dari Vektorisasi Data
\par Penjelasan : Pembagian dan pemecahan data, kemudian dilakukan perhitungan. Vektorisasi juga dapat dimaksudkan dengan setiap data yang mungkin dipetakan ke integer tertentu. jika kita memiliki array yang cukup besar maka setiap kata / data cocok dengan slot unik dalam array (nilai pada indeks adalah nomor satu kali kata itu muncul).
\par Array angka floating point ( Mewakili data ) :
\begin{itemize}
\item teks
\item audio
\item gambar
\end{itemize}
\par Contoh : -[1.0, 0.0, 1.0, 0.5]
\par
\end{itemize}
\par
\par
\item Pengertian Bag Of Words Dan Ilustrasi Gambar
\begin{itemize}
\item  Pengertian Bag Of Words
\par bag-of-words adalah representasi penyederhanaan yang digunakan dalam pemrosesan bahasa alami dan pengambilan informasi. Model bag-of-words sederhana untuk dipahami dan diterapkan dan telah melihat kesuksesan besar dalam masalah seperti pemodelan bahasa dan klasifikasi dokumen ( penyelesaian dll ).
\par
\par
\item Ilustrasi Gambar
\par Penjelasan : Berdasarkan pengertian diatas, ada beberapa contoh yang bisa diterapkan. Untuk salah satu contoh dari Bag Of Words sendiri dapat diliat pada gambar berikut \ref{bag-fadila}.
\begin{figure}[!hbtp]
\centering
\includegraphics[scale=0.3]{figures/bag-fadila.jpg}
\caption{bag-fadila}
\label{bag-fadila}
\end{figure}
\par
\end{itemize}
\par
\par
\item Pengertian TF-IDF Dan Ilustrasi Gambar
\begin{itemize}
\item  Pengertian TF-IDF
\par TF-IDF atau TFIDF, adalah kependekan dari istilah frekuensi dokumen terbalik, dimana merupakan statistik numerik yang dimaksudkan untuk mencerminkan betapa pentingnya sebuah kata untuk sebuah dokumen dalam kumpulan atau kumpulan.
\par Nilai tf-idf meningkat secara proporsional dengan berapa kali sebuah kata muncul dalam dokumen dan diimbangi dengan jumlah dokumen dalam korpus yang mengandung kata, yang membantu menyesuaikan fakta bahwa beberapa kata muncul lebih sering secara umum
\item Ilustrasi Gambar
\par Penjelasan : Berdasarkan pengertian diatas, ada beberapa contoh yang bisa diterapkan. Untuk salah satu contoh dari TF-IDF sendiri dapat diliat pada gambar-gambar berikut \ref{tf2-fadila}.
\begin{figure}[!hbtp]
\centering
\includegraphics[scale=0.4]{figures/tf2-fadila.jpg}
\caption{tf2-fadila}
\label{tf2-fadila}
\end{figure}
\par
\par
\end{itemize}
\par
\par

\end{enumerate}




\section{Rahmi Roza-1164085}
\subsection{Teori}
Penjelasan Tugas Harian 7 ( No 1-6 )
\begin{enumerate}
\item Pengertian Klasifikasi Teks Dan Ilustrasi Gambar
\begin{itemize}
\item Pengertian Klasifikasi Teks
\par Klasifikasi teks adalah proses proses pengelompokkan bendaberdasarkan ciri-ciri persamaan dan perbedan dengan pemberian tag atau kategori ke teks sesuai dengan isinya. 
\par
\item Ilustrasi Gambar
\par Penjelasan : Berdasarkan pengertian diatas, ada beberapa contoh yang bisa diterapkan. Untuk salah satu contoh dari klasifikasi data sendiri dapat diliat pada gambar berikut \ref{ktRoza}.
\begin{figure}[!hbtp]
\centering
\includegraphics[scale=0.3]{figures/ktRoza.png}
\caption{Klasifikasi Teks Roza}
\label{text-fadila}
\end{figure}
\par
\end{itemize}
\par
\par
\item Mengapa Klasifikasi Bunga Tidak Bisa Menggunakan Machine Learning Dan Ilustrasi Gambar
\begin{itemize}
\item  Mengapa Klasifikasi Bunga Tidak Bisa Menggunakan Machine Learning
\par Penjelasan : Klasifikasi bunga tidak bisa digunakan pada machine learning karena terdapat masalah input yang sama dan output yang berbeda. Output atau error disebut noise.
\par
\item Ilustrasi Gambar
\par Penjelasan : Berdasarkan pengertian diatas, ada beberapa contoh yang bisa diterapkan. Untuk salah satu contoh dari klasifikasi bunga sendiri dapat diliat pada gambar berikut \ref{flowerRoza}.
\begin{figure}[!hbtp]
\centering
\includegraphics[scale=0.3]{figures/flowerRoza.jpg}
\caption{Klasifikasi Bunga Roza}
\label{text-fadila}
\end{figure}
\par
\end{itemize}
\par
\par
\item Teknik Pembelajaran Mesin Pada Teks Pada Kata-Kata Yang Digunakan Di Youtube Dan Ilustrasi Gambar
\begin{itemize}
\item  Teknik Pembelajaran Mesin Pada Teks Pada Kata-Kata Yang Digunakan Youtube
\par Penjelasan : Penggunaan Machine Learning pada Youtube yaitu contohnya pada saat kita melakukan searching vidio atau pun bahan lainnya di youtube maka yang akan ditampilkan adalah vidio dari keyword yang kita ketikkan di kotak pencarian. Dengan kata lain Youtube akan mempfilter vidio dengan keywoard yang telah digunakan. Dan pada saat kita menonton Youtube pada bagian sebelah kanan tampilan youtubr terdapat vidio yang berkaitan dengan keyworad yang kita cari tadi. Dimana disitulah Machine Learning pada Youtube menyimpan data.
\par
\item Ilustrasi Gambar
\par Penjelasan : Berdasarkan pengertian diatas, ada beberapa contoh yang bisa diterapkan. Untuk salah satu contoh dari Mesin Teks Youtube sendiri dapat diliat pada gambar berikut \ref{YoutubeRoza}.
\begin{figure}[!hbtp]
\centering
\includegraphics[scale=0.4]{figures/YoutubeRoza.png}
\caption{Youtube Roza}
\label{text-fadila}
\end{figure}
\par
\end{itemize}
\par
\par
\item Vektorisasi Data
\begin{itemize}
\item Maksud Dari Vektorisasi Data
\par Penjelasan : Pembagian dan pemecahan data, dan kemudian dilakukan perhitungan datanya. Vektorisasi juga dapat dimaksudkan dengan setiap data yang mungkin dipetakan ke integer tertentu. Yang mana data tersebut dalam bentuk data vektor diperoleh dalam bentuk koordinat titik yang menampilkan, menempatkan dan menyimpan data spasial dengan menggunakan titik, garis atau area (poligon). 
\par
\end{itemize}

\item Pengertian Bag Of Words Dan Ilustrasi Gambar
\begin{itemize}
\item  Pengertian Bag Of Words
\par Bag Of-Words adalah sebuah konsep yang diambil dari analisis teks yaitu mempresentasikan dokumen sebagai sebuah kantung informasi-infromasi penting tanpa mengurutkan setiap katanya.
\par
\item Ilustrasi Gambar
\par Penjelasan : Berdasarkan pengertian diatas, ada beberapa contoh yang bisa diterapkan. Untuk salah satu contoh dari Bag Of Words sendiri dapat diliat pada gambar berikut \ref{BagofwordsRoza.jpg}.
\begin{figure}[!hbtp]
\centering
\includegraphics[scale=0.2]{figures/BagofwordsRoza.jpg}
\caption{Bag of Words Roza}
\label{bag-fadila}
\end{figure}
\par
\end{itemize}
\par
\par

\item Pengertian TF-IDF Dan Ilustrasi Gambar
\begin{itemize}
\item  Pengertian TF-IDF
\par TF-IDF  (Term Frequency – Inverse Document Frequency) adalah  sebuah algoritma  adalah salah satu algoritma yang dapat digunakan untuk menganalisa hubungan antara sebuah frase/kalimat dengan sekumpulan dokumen. 
\par Inti utama dari algoritma ini adalah melakukan perhitungan nilai TF dan nilai IDF dari sebuah setiap kata kunci terhadap masing-masing dokumen. Nilai TF dihitung dengan rumus TF = jumlah frekuensi kata terpilih / jumlah kata dan nilai IDF dihitung dengan rumus IDF = log(jumlah dokumen / jumlah frekuensi kata terpilih). Selanjutnya adalah melakukan perkalian antara nilai TF dan IDF untuk mendapatkan jawaban akhir.
\item Ilustrasi Gambar
\par Penjelasan : Berdasarkan pengertian diatas, ada beberapa contoh yang bisa diterapkan. Untuk salah satu contoh dari TF-IDF sendiri dapat diliat pada gambar berikut \ref{TFIDFroza.jpeg}.
\begin{figure}[!hbtp]
\centering
\includegraphics[scale=0.2]{figures/TFIDFroza.jpeg}
\caption{TF-ID Rroza}
\label{tf-fadila}
\end{figure}
\par
\end{itemize}
\par
\par

\end{enumerate}

\subsection{Praktek}